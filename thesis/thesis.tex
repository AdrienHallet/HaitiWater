\documentclass{eplmastersthesis_FR}

\title{HaïtiWater}
\subtitle{Développement d'une application web pour gérer la distribution de l'eau en Haïti}
\author{Céline \textsc{Deknop}}
\secondauthor{Adrien \textsc{Hallet}}
\thirdauthor{Sébastien \textsc{Strebelle}} % Handcrafted third author :D
\speciality{Sciences Informatiques}
%\options{Option(s)} % If required by program commission mention options
\supervisor{Kim \textsc{Mens}}
\cosupervisor{Sandra \textsc{Soares Frazao}}
\readerone{Benoît \textsc{Duhoux}}
\readertwo{To be \textsc{Determined}}
%\readerthree{Firstname \textsc{Lastname}}
\years{2018-2019}

\usepackage{hyperref}

\begin{document}

	\frontpage
	\tableofcontents

	\setlength{\parskip}{1.5em plus1em minus1em}

	% Total des pages : entre 49 et 75 d'après nos estimations.

	\chapter*{Abstract}
	\addcontentsline{toc}{chapter}{Abstract}

		Page : 1

	\chapter{Introduction}

		% Pages : 2 à 3

		\subsection*{Contexte}

			Ce mémoire se place dans le cadre d'un projet de développement lancé par l'ONG Protos\footnote{\href{https://www.protos.ngo/fr/}{www.protos.ngo}}. Protos est une ONG qui a pour but d'améliorer l'accès à l'eau potable afin d'aider le développement de plusieurs pays du monde. Un des pays dans lesquels Protos est engagé est Haïti.

			Suite à plusieurs problèmes naturels et politiques récents, la situation de l'accès à l'eau potable  est particulièrement compliquée à Haïti. Le séïsme de 2010 a laissé le pays en ruine et a détruit beaucoup d'installations, y compris de distribution d'eau. Le chaos politique qui en a suivi n'a pas aidé à la reconstruction de ces installations et le peuple ne peut toujours pas réellement compter sur le pouvoir politique pour assurer la distribution de l'eau. C'est --- entre autre --- pourquoi l'ONG Protos est active dans le pays.
			% TODO trop chargé politiquement ?

			Protos a contacté l'UCL afin de réaliser un projet de développement en Haïti. Le but de ce projet est de réaliser un système de gestion pour la distribution de l'eau potable. Ce projet est prévu sur trois ans, une première année pour la conception d'un prototype, une deuxième année pour finaliser le système et une troisième année pour déployer le système sur place. Ce mémoire consiste en la première partie de ce projet.

			% Notre mémoire est un mémoire de développement d'une application web utilisée en Haïti.
			% Contactés par l'ONG protos
			% Situation d'haiti problématique (manque de moyens financiers et humains)

		\subsection*{Problème}

			La structure hiérarchique des différents acteurs de la distibution de l'eau, gouvernementaux ou non, est assez complexe. Cependant, ces acteurs doivent se coordonner pour arriver à une gestion efficace des ressources. Dans ce mémoire, nous nous concentrons sur quelques problèmes rencontrés par ces acteurs afin d'améliorer leur coordination.

			Premièrement, nous essayons d'aider les problèmes d'organisation entre les différents acteurs. Deuxièmement, nous travaillons sur l'amélioration de la communication entre eux. Troisièmement, nous cherchons à améliorer la gestion des données de la distribution de l'eau. % blbl, TODO make this better

			% Le problème que nous, en tant que mémoire, on traite : organisation, communication, gestion des données

		\subsection*{Motivation}

			Ce mémoire se place dans le cadre de nos études de Master en Sciences Informatiques. Il forme notre travail de fin d'études. Le but de ce travail est de compléter nos études, et de nous permettre de mettre tous nos apprentissages en application dans un projet à grande échelle.

			Il s'agit de plus dans ce cas ci d'un projet réel, avec de véritables acteurs de terrains et ayant un vrai objectif d'utilisation à long terme. Cela comporte plein de problèmatiques nouvelles pour nous, qui ne sont pas abordées dans le reste de nos études.

			Ce mémoire nous permet d'avoir une première expérience de développement d'une application web, partant de rien sauf des attentes de nos clients, et intégrant toutes les parties et les étapes de ce développement.

			Ce mémoire a aussi pour but d'être utile pour Haïti. Nous espérons que l'application développée dans le cadre de ce mémoire pourra être utilisée sur place et pourra avoir un réel impact, aidant la distribution de l'eau à Haïti et son développement.

			% Pourquoi on le fait, car l'ONG a contacté l'UCL, on développe l'application pour que Haiti se développe
			% Nos motivations personnelles ; full stack dev ops

		\subsection*{Objectif}

			Le but final de ce mémoire est de livrer une application à l'ONG Protos et aux acteurs de la distribution de l'eau à Haïti. Même si cette application ne sera peut-être pas directement utilisée sur le terrain, nous espérons qu'elle poura aider ces acteurs à créer une base applicative qu'ils pourront maintenir et développer afin de répondre à leurs besoins.

      % Leur créer une application qui leur permette :
       % Soit de construire une base applicative sur laquelle ils pourront se développer
       % Soit de se rendre compte de ce qu'une solution logicielle peut leur apporter, et donc de repenser leurs besoins.
      % On est une application de transition, pas une multinationale (lien groupe Saur)

		\subsection*{Approche}
      % Longue phase de requirements, d'abord pour voir ce qui se fait dans le monde
      % Seconde phase des requirements ; situation actuelle en Haïti, solutions mises en place, leurs limites
      % Design ; nos propositions
      % Implémentation ; réalisation
      % Validation (?) => mise en attente jusqu'à ce qu'on le fasse
		\subsection*{Contribution}
      % Par notre biais, protos/haiti a eu premier contact avec le monde du developpement logiciel
      % Proposer une application pour la gestion de l'eau potable en haiti, qui soit une base de travail pour la suite de leur développement national.
		\subsection*{Plan}
      % Expliquer ce qu'on va expliquer.

	\chapter{Contexte}

		Total des pages : 6 à 10

		\section{Situation de l'eau à Haïti}

			Pages : 2 à 3

			\subsection*{Problèmes naturels}
			\subsection*{Problèmes politiques}
			\subsection*{Problèmes sociaux}
			\subsection*{Problèmes organisationnels}

		\section{Gestion actuelle}

			Pages : 2 à 3

			\subsection*{Organisation du pays}
			\subsection*{Structure organisationnelle}
			\subsection*{Procédures actuelles}

		\section{Comparaison avec d'autres pays}

			Pages : 1 à 2

			\subsection*{Gestion de l'eau en belgique}
			\subsection*{Visite d'un centre opérationnel en France}

		\section{Comparaison avec des outils existants}

			Pages : 1 à 2

			% TODO find them


	\chapter{Approche}

		Total des pages : 3 à 6

		\section{Organisation du travail}

			Pages : 1 à 2

			\subsection*{Planning}
			\subsection*{Réunions}

		\section{Répartition des tâches}

			Pages : 1 à 2

			\subsection*{Répartition des tâches quotidiennes}
			\subsection*{Répartition de l'analyse}
			\subsection*{Répartition de l'implémentation}
			\subsection*{Répartition de l'écriture}

		\section{Méthodologie}

			Pages : 1 à 2

			\subsection*{Méthodologie agile}
			\subsection*{Phases de développement}

	\chapter{Analyse des besoins}

		Total des pages : 6 à 10

		\section{Besoins fonctionnels}

			Pages : 1 à 2

			\subsection*{Gestion des données}
			\subsection*{Simplification des procédures}

		\section{Besoins non-fonctionnels}

			Pages : 1 à 2

			\subsection*{Sécurité des données}
			\subsection*{Connexions lentes et peu fiables}

		\section{Cahier des charges}

			Pages : 2 à 3

			\subsection*{Complet en annexe ?}

		\section{Structure des données}

			Pages : 2 à 3

			\subsection*{Complet en annexe ?}

	\chapter{Implémentation}

		Total des pages : 16 à 22

		La structure proposée n'est par conséquent pas définitive et dépendra des résultats obtenus.

		\section{Choix technologiques}

			Pages : 3 à 4

			\subsection*{Web}
			\subsection*{Python}
			\subsection*{Django}
			\subsection*{PostGIS}
			\subsection*{DataTables}
			\subsection*{ChartJS}

		\section{Structure hiérarchique des utilisateurs}

			Pages : 1 à 2

			\subsection*{Structure}
			\subsection*{Permissions}

		\section{Interface utilisateur}

			Pages : 2 à 3

			\subsection*{Référence en annexe ?}

		\section{Procédure d'utilisation}

			Pages : 2 à 3

			\subsection*{Référence en annexe ?}

		\section{Client}

			Pages : 4 à 5
			\subsection*{Modularité et responsiveness}
			\subsection*{Gabarits}
			\subsection*{Accessibilité hors-ligne}
			\subsection*{...}

		\section{Serveur}

			Pages : 4 à 5
			\subsection*{Authentification}
			\subsection*{Requêtes}
			\subsection*{...}

	\chapter{Validation}

		Total des pages : 8 à 11

		\section{Performances}

			Pages : 3 à 4

			\subsection*{Temps}
			\subsection*{Poids}

		\section{Vérifications automatiques}

			Pages : 2 à 3

			\subsection*{Tests unitaires}
			\subsection*{Tests fonctionnels}

		\section{Vérifications utilisateurs réels}

			Pages : 3 à 4

			\subsection*{Méthodologie}
			\subsection*{Résultats obtenus}
			\subsection*{Modifications apportées}

	\chapter{Améliorations futures}

		Total des pages : 4 à 7

		\section{Suite du projet}

			Pages : 2 à 3

		\section{Défis rencontrés}

			Pages : 1 à 2

		\section{Propositions}

			Pages : 1 à 2

	\chapter{Conclusion}

		Pages : 1 à 2

	\chapter*{Bibliographie}
	\addcontentsline{toc}{chapter}{Bibliographie}

		Pages : 2 à 3

	\appendix

	\chapter{Cahier des charges complet}

		Pages : beaucoup

	\chapter{Base de données}

		Pages : beaucoup

	\chapter{Wireframes}

		Pages : beaucoup

	\chapter{Diagrammes d'activité}

		Pages : beaucoup

	\setlength{\parskip}{0em}
	\backcoverpage

\end{document}
