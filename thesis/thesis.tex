\documentclass{eplmastersthesis_FR}

\title{HaitiWater}
\subtitle{Développement d'une application web pour gérer la distribution de l'eau en Haïti}
\author{Céline \textsc{Deknop}}
\secondauthor{Adrien \textsc{Hallet}}
\thirdauthor{Sébastien \textsc{Strebelle}} % Handcrafted third author :D
\speciality{Sciences Informatiques}
%\options{Option(s)} % If required by program commission mention options
\supervisor{Kim \textsc{Mens}}
\cosupervisor{Sandra \textsc{Soares Frazao}}
\readerone{Benoit \textsc{Duhoux}}
\readertwo{To be \textsc{Determined}}
%\readerthree{Firstname \textsc{Lastname}}
\years{2018-2019}

\begin{document}

	\maketitle
	\thispagestyle{empty}
	% To suppress header and footer on the back of the cover page

	Total des pages : entre 45 et 73 d'après nos estimations. Donc on devrait pouvoir respecter le nombre de page à la version finale.

	\chapter*{Abstract}


		Page : 1

	\chapter{Introduction}

		Pages : 2 à 3

		\section*{Contexte}

		\section*{Problème}

		\section*{Motivation}

		\section*{Objectif}

		\section*{Approche}

		\section*{Contribution}

		\section*{Plan}

	\chapter{Contexte}

		\section{Situation de l'eau à Haïti}

			Pages : 2 à 3

			%\subsection*{subsection name}


		\section{Gestion actuelle}

			Pages : 3 à 4

		\section{Comparaison avec d'autres pays}

			Pages : 1 à 2

	\chapter{Approche}

		\section{Organisation du travail}

			Pages : 2 à 3

		\section{Répartition des tâches}

			Pages : 1 à 2


	\chapter{Analyse des besoins}

		\section{Besoins des clients}

			Pages : 1 à 2

		\section{Cahier des charges}

			Pages : 3 à 4 (éventuellement référence annexe)

		\section{Structure des données}

			Pages : 2 à 3

	\chapter{Implémentation}
		Notez que le contenu de ce chapitre va dépendre en grande partie des résultats obtenus. La structure proposée n'est par conséquent pas définitive.

		\section{Choix technologiques}

			Pages : 2 à 3

		\section{Structure hiérarchique des utilisateurs}

			Pages : 1

		\section{Interface utilisateur}

			Pages : 3 à 4 (ref annexe)

		\section{Procédure d'utilisation}

			Pages : 2 à 3

		\section{Structure de l'application}

			\subsection*{Utilisateurs}

		\section{Client}

			Pages : 3 à 4
			\subsection*{Modularité et responsiveness}
			\subsection*{Gabarits}
			\subsection*{Accessibilité hors-ligne}

		\section{Serveur}

			Pages : 2 à 10
			\subsection*{Authentification}
			\subsection*{Requêtes}

		\section{Défis rencontrés}

			Pages : 1 à 2

	\chapter{Validation}

		\section{Performances}

			Pages : 3 à 4

			\subsection*{Temps}
			\subsection*{Poids}

		\section{Vérifications automatiques}

			Pages : 1 à 2

			\subsection*{Tests unitaires}
			\subsection*{Tests fonctionnels}

		\section{Vérifications utilisateurs réels}

			Pages : 2 à 3

			\subsection*{Méthodologie}
			\subsection*{Résultats obtenus}
			\subsection*{Modifications apportées}

	\chapter{Améliorations futures}

		\section{Suite du projet}

			Pages : 2 à 3

		\section{Propositions}

			Pages : 1 à 2

	\chapter{Conclusion}

		Pages : 2

	\chapter*{Bibliographie}

		Pages : 2 à 3

	\appendix

	\chapter{Cahier des charges complet}

		Pages : beaucoup

	\chapter{Wireframes}

		Pages : beaucoup


	\backcoverpage

\end{document}
