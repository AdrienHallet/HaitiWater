\documentclass{eplmastersthesis_FR}

\title{HaitiWater}
\subtitle{Développement d'une application web pour gérer la distribution de l'eau en Haïti}
\author{Céline \textsc{Deknop}}
\secondauthor{Adrien \textsc{Hallet}}
\thirdauthor{Sébastien \textsc{Strebelle}} % Handcrafted third author :D
\speciality{Sciences Informatiques}
%\options{Option(s)} % If required by program commission mention options
\supervisor{Kim \textsc{Mens}}
\cosupervisor{Sandra \textsc{Soares Frazao}}
\readerone{Benoît \textsc{Duhoux}}
\readertwo{To be \textsc{Determined}}
%\readerthree{Firstname \textsc{Lastname}}
\years{2018-2019}

\begin{document}

	\maketitle
	\thispagestyle{empty}
	% To suppress header and footer on the back of the cover page

	Total des pages : entre 49 et 75 d'après nos estimations.

	\chapter*{Abstract}

		Page : 1

	\chapter{Introduction}

		Pages : 2 à 3

		\subsection*{Contexte}
		\subsection*{Problème}
		\subsection*{Motivation}
		\subsection*{Objectif}
		\subsection*{Approche}
		\subsection*{Contribution}
		\subsection*{Plan}

	\chapter{Contexte}

		Total des pages : 6 à 10

		\section{Situation de l'eau à Haïti}

			Pages : 2 à 3

			\subsection*{Problèmes naturels}
			\subsection*{Problèmes politiques}
			\subsection*{Problèmes sociaux}
			\subsection*{Problèmes organisationnels}

		\section{Gestion actuelle}

			Pages : 2 à 3

			\subsection*{Organisation du pays}
			\subsection*{Structure organisationnelle}
			\subsection*{Procédures actuelles}

		\section{Comparaison avec d'autres pays}

			Pages : 1 à 2

			\subsection*{Gestion de l'eau en belgique}
			\subsection*{Visite d'un centre opérationnel en France}

		\section{Comparaison avec des outils existants}

			Pages : 1 à 2

			% TODO find them


	\chapter{Approche}

		Total des pages : 3 à 6

		\section{Organisation du travail}

			Pages : 1 à 2

			\subsection*{Planning}
			\subsection*{Réunions}

		\section{Répartition des tâches}

			Pages : 1 à 2

			\subsection*{Répartition des tâches quotidiennes}
			\subsection*{Répartition de l'analyse}
			\subsection*{Répartition de l'implémentation}
			\subsection*{Répartition de l'écriture}

		\section{Méthodologie}

			Pages : 1 à 2

			\subsection*{Méthodologie agile}
			\subsection*{Phases de développement}

	\chapter{Analyse des besoins}

		Total des pages : 6 à 10

		\section{Besoins fonctionnels}

			Pages : 1 à 2

			\subsection*{Gestion des données}
			\subsection*{Simplification des procédures}

		\section{Besoins non-fonctionnels}

			Pages : 1 à 2

			\subsection*{Sécurité des données}
			\subsection*{Connexions lentes et peu fiables}

		\section{Cahier des charges}

			Pages : 2 à 3

			\subsection*{Complet en annexe ?}

		\section{Structure des données}

			Pages : 2 à 3

			\subsection*{Complet en annexe ?}

	\chapter{Implémentation}

		Total des pages : 16 à 22

		La structure proposée n'est par conséquent pas définitive et dépendra des résultats obtenus.

		\section{Choix technologiques}

			Pages : 3 à 4

			\subsection*{Web}
			\subsection*{Python}
			\subsection*{Django}
			\subsection*{PostGIS}
			\subsection*{DataTables}
			\subsection*{ChartJS}

		\section{Structure hiérarchique des utilisateurs}

			Pages : 1 à 2

			\subsection*{Structure}
			\subsection*{Permissions}

		\section{Interface utilisateur}

			Pages : 2 à 3

			\subsection*{Référence en annexe ?}

		\section{Procédure d'utilisation}

			Pages : 2 à 3

			\subsection*{Référence en annexe ?}

		\section{Client}

			Pages : 4 à 5
			\subsection*{Modularité et responsiveness}
			\subsection*{Gabarits}
			\subsection*{Accessibilité hors-ligne}
			\subsection*{...}

		\section{Serveur}

			Pages : 4 à 5
			\subsection*{Authentification}
			\subsection*{Requêtes}
			\subsection*{...}

	\chapter{Validation}

		Total des pages : 8 à 11

		\section{Performances}

			Pages : 3 à 4

			\subsection*{Temps}
			\subsection*{Poids}

		\section{Vérifications automatiques}

			Pages : 2 à 3

			\subsection*{Tests unitaires}
			\subsection*{Tests fonctionnels}

		\section{Vérifications utilisateurs réels}

			Pages : 3 à 4

			\subsection*{Méthodologie}
			\subsection*{Résultats obtenus}
			\subsection*{Modifications apportées}

	\chapter{Améliorations futures}

		Total des pages : 4 à 7

		\section{Suite du projet}

			Pages : 2 à 3

		\section{Défis rencontrés}

			Pages : 1 à 2

		\section{Propositions}

			Pages : 1 à 2

	\chapter{Conclusion}

		Pages : 1 à 2

	\chapter*{Bibliographie}

		Pages : 2 à 3

	\appendix

	\chapter{Cahier des charges complet}

		Pages : beaucoup

	\chapter{Base de données}

		Pages : beaucoup

	\chapter{Wireframes}

		Pages : beaucoup

	\chapter{Diagrammes d'activité}

		Pages : beaucoup

	\backcoverpage

\end{document}
