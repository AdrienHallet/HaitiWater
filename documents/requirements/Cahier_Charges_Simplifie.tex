\documentclass[a4paper, 11pt]{article}
\usepackage[french]{babel}
\usepackage[utf8]{inputenc}
\usepackage[T1]{fontenc}
\usepackage{datetime}
\usepackage{float}
\usepackage{eurosym}
\usepackage{comment} % enables the use of multi-line comments (\ifx \fi)
\usepackage{graphicx}

\usepackage{fullpage} % changes the margin
\begin{document}
%Header-Make sure you update this information!!!!
\noindent
\large\textbf{Cahier des charges résumé} \hfill \textbf{Distribution de l'eau en Haïti} \\
\normalsize Deknop Céline \hfill Université Catholique de Louvain \\
Hallet Adrien \hfill \today \\
Strebelle Sébastien

\section*{Abstract}
Cahier des charges simplifié à destination des intervenants extérieurs à l'équipe de développement. Ce document vise à harmoniser les vues de tous les acteurs impliqués par le projet d'application web pour la distribution d'eau en Haïti.
\hrule

\section{Synthèse}
  L'application se dessine comme une version améliorée des documents Excel fournis. Les améliorations sont :
  \begin{description}
    \item[Performances] - Permettre l'utilisation concurrente de plusieurs utilisateurs sur de plus grandes données (ayant pour but final une utilisation nationale si l'application satisfait les besoins).
    \item[Intuitivité] - Permettre une utilisation plus facile du système grâce à une interface plus simple que le tableur, des formulaires, une validation des données pour empêcher les erreurs communes.
    \item[Graphiques] - Grâce à des informations graphiques automatiques, encourager l'utilisation de l'application. Les informations graphiques permettront une compréhension plus rapide des données.
    \item[Regroupement] - Regrouper les données similaires sur les mêmes pages et ainsi diminuer le nombre d'onglets de l'application par rapport au fichier Excel. Cela permet une utilisation facilitée de l'application et de trouver une donnée plus rapidement.
  \end{description}

\section{Utilisateurs}
\begin{description}
  \item[Administrateur] ~ \\
  Initialement Protos, l'administrateur sera le gestionnaire de la totalité du réseau. Il peut consulter et modifier tous les utilisateurs et le réseau de distribution de la totalité du pays.

  \item[Gestion de Zone] ~ \\
  Le gestionnaire de Zone possède les droits d'administration sur une zone géographique déterminée. Typiquement les CAEPA. Le gestionnaire de zone peut consulter tous les utilisateurs et le réseau de distribution dans sa zone.

  \item[Gestion de Point d'eau] ~ \\
  Le gestionnaire du point d'eau possède les droits d'administration sur son point d'eau. Il peut utiliser le système pour envoyer les informations relatives à son point d'eau uniquement.

  \item[Consommateur](\emph{Type d'utilisateur futur pour le volet "CitizenScience"}) \\
  Le consommateur est un utilisateur du réseau de distribution. Le consommateur peut utiliser le système pour rapporter des problèmes, informer le réseau et ses gestionnaires.
\end{description}

\section{Données}
Les données sont réparties en catégories. Nous listons ci-dessous les données générales qui seront enregistrées dans le système. La totalité des champs est disponible dans les fichiers Excel actuellement utilisés.
\begin{description}
  \item[Gestionnaires]
    Enregistre les données des gestionnaires du réseau (TEPAC - Technicien en Eau potable et Assainissement au niveau des Communes).

  \item[Utilisateurs] ~ \\
    Enregistre les informations légales d'un utilisateur pour l'identifier dans le réseau. On enregistre également son point d'alimentation dans le réseau.

  \item[Réseau] ~ \\
    Le réseau sera enregistré dans le système. On considère toutes les informations disponibles sur le réseau (fontaines, kiosques, ...) et des extensions pour pouvoir créer des analyses (débits, qualité, ...) si ces données deviennent disponibles.

    A terme, ces données seront utilisées en parallèle avec des données géographiques pour une représentation visuelle (dynamique) du réseau.

  \item[Budgets] ~ \\
    Les informations comptables (budgets, investissements) sur les différents projets et périodes comptables.

  \item[Facturation] ~ \\
    Les informations de redevance pour le réseau. Ces informations permettront de créer les factures et leur suivi.


\end{description}

\section{Fonctionnalités}
  \subsection{Base}
  \begin{enumerate}
    \item Faciliter l'enregistrement (via formulaire) des données nécessaires à la gestion du réseau voir section ci-dessus.
    \item Permettre la visualisation (graphique et numérique) des données.
    \item Permettre la visualisation (graphique et numérique) des aggrégations de données (e.g.: performances, recourvement, ...). Destiné aux gestionnaires de réseau (zone et général).
    \item Faciliter le recouvrement des factures en automatisant leur génération et facilitant leur consultation.
    \item Permettre la visualisation graphique (Système d'Information Géographique) du réseau avec interactions dynamiques (e.g.: cliquer sur les fontaines pour en voir les informations)
    \item Faciliter la gestion des budgets par une interface moins numérique et plus visuelle. A définir selon les besoins réels.
    \item Système de "tickets" (messagerie interne) permettant de rapporter des informations entre les différents niveaux hiérarchiques du système (DINEPA, CAPEA, TEPAC).
  \end{enumerate}
  \subsection{Extensions}
  \begin{enumerate}
    \item CitizenScience ; permettre aux usagers d'envoyer des informations aux gestionnaires (e.g.: problèmes)
    \item Permettre aux utilisateurs de consulter l'état de leur consommation et du réseau (proposition)
    \item Analyse des données en vue d'optimiser le réseau
    \item Utilisation sans connexion Internet
  \end{enumerate}

\section{Caractéristiques}
\begin{enumerate}
  \item Système performant pour permettre une utilisation dans les conditions haïtiennes.
  \item Sécurité des données.
  \item Fonctionnement sur périphérique mobile (tablettes et smartphones).
  \item Licenses libres pour les composantes technologiques du système.
  \item Utilisation de technologies courantes/répandues dans le système afin de permettre une meilleure intégration des futurs développeurs dans le système.
  \item Présence d'une documentation qualitative pour les développeurs et les utilisateurs du système.
  \item Permettre une mise à l'échelle de l'application au niveau national
\end{enumerate}

\section{User stories}
Les user stories suivantes sont hiérarchiques, c'est-à-dire que les droits des supérieurs hiérarchiques englobent ceux de leur subordonnés.
\subsection{En tant qu'utilisateur,}
\begin{itemize}
  \item Je dois me connecter sur l'application pour voir les outils qui me sont accessibles.
  \item Je peux accéder à mes permissions pour consulter les points d'eau qui me sont assignés.
  \item Je peux effectuer mon travail hors ligne, pour l'envoyer plus tard quand j'aurai du réseau % <- Impossible avec Django, on laisserait pas ça de côté pour le moment ? ça fait déjà beaucoup de travail et ils peuvent toujours noter sur leur pc et uploader quand ils ont une connexion
\end{itemize}

\subsection{En tant que gestionnaire de fontaine (Comité de Fontaine),}
\begin{itemize}
  \item Je peux signaler un problème (matériel, qualité de l'eau, ...) à mon point d'eau pour que le technicien et mes supérieurs en soient informés.
  \item Je peux modifier un problème pour donner de nouvelles informations ou en modifier l'urgence.
  \item Je peux clôturer un problème pour indiquer qu'il est résolu.
  \item Je dois rentrer les données d'utilisation mensuelles correspondant à mon point d'eau anfin de les utiliser dans la gestion du réseau.
\end{itemize}

% Ils doivent également gérer les conflits, est-ce qu'on veut faire un pannel pour ça ou de la doc ou whatever ?

\subsection{En tant que gestionnaire de groupe de points d'eau,}
\begin{itemize}
  \item Je peux voir toutes les fontaines dans mon groupe pour pouvoir les sélectionner.
  \item Je dois sélectionner un point d'eau pour accéder à ses données.
  \item Je peux consulter toutes les données entrées par un gestionnaire de point d'eau dans ma zone afin de m'en informer.
  \item Je peux consulter des graphiques générés automatiquement pour évaluer l'état du réseau, des utilisateurs et factures dans ma zone.
  \item Je peux consulter les problèmes déclarés pour voir quels points d'eau sont en difficulté.
  \item Je peux assigner un technicien à un problème particulier pour le prioriser.
  \item Je peux modifier l'état d'un problème pour en marquer la résolution ou la prise en charge.
  \item Je peux aggréger les rapports mensuels pour préparer la réunion annuelle.
\end{itemize}

\subsection{En tant qu'administrateur (gestionnaire général),}
\begin{itemize}
  \item Je suis un gestionnaire de groupe qui a accès à la totalité des données pour consulter toutes les données.
  \item Je peux ajouter un utilisateur afin d'autoriser plus de personnes à travailler avec l'application.
  \item Je peux supprimer un utilisateur lorsque celui-ci n'est plus nécessaire.
  \item Je peux créer un groupe d'utilisateur avec des permissions spécifiques sur les points d'eau afin de ne pas devoir les assigner à chaque utilisateur individuellement.
  \item Je peux assigner un point d'eau à un groupe pour que ce groupe gère le point d'eau.
  \item Je peux retirer un point d'eau à un groupe pour que ce groupe ne gère plus le point d'eau.
  \item Je peux assigner un utilisateur à un groupe pour que cet utilisateur dispose des permissions héritées du groupe.
  \item Je peux retirer un utilisateur d'un groupe pour que cet utilisateur n'ait plus les permissions héritées du groupe.

\end{itemize}

\subsection{En tant que technicien/plombier,}
\begin{itemize}
  \item Je peux consulter les problèmes du réseau pour déterminer la nécessité d'intervenir ou non.
  \item Je peux consulter les problèmes priorisés par le gestionnaire de zone pour voir où une intervention est requise.
  \item Je peux répondre aux problèmes afin de demander des précisions ou donner des solutions.
  \item Je peux modifier l'état d'un problème afin d'en indiquer la prise en charge ou résolution.
\end{itemize}

\subsection{À préciser}
\begin{itemize}
  \item En tant qu'utilisateur, je souhaite avoir une vue générale comprenant les informations essentielles me concernant
\end{itemize}

\emph{Commentaire général : ici je me suis basée sur le document qu'on a reçu concernant Poste-Métier (si je me souviens bien). Le fonctionnement en comité est propre à cet endroit, mais puisqu'il semble bien fonctionner je me suis dit que, puisqu'il faut se baser sur qqch, autant le faire sur ça}


\end{document}
