\documentclass[a4paper, 11pt]{article}
\usepackage[french]{babel}
\usepackage[utf8]{inputenc}
\usepackage[T1]{fontenc}
\usepackage{datetime}
\usepackage{float}
\usepackage{eurosym}
\usepackage{comment} % enables the use of multi-line comments (\ifx \fi)
\usepackage{graphicx}

\usepackage{fullpage} % changes the margin
\begin{document}
%Header-Make sure you update this information!!!!
\noindent
\large\textbf{Synthèse "Analyse Contextuelle Commune"} \hfill \textbf{Distribution de l'eau en Haïti} \\
\normalsize Deknop Céline \hfill Université Catholique de Louvain \\
Hallet Adrien \hfill \today \\
Strebelle Sébastien

\section*{Abstract}
Synthèse du document \textit{Analyse Commune de Contexte Haïti} envoyé par Protos le 05 septembre 2018. Ce document présente le contexte général d'Haïti sous tous les angles pertinents (économie, écologie, enseignement, ...). Le document en lui-même date d'octobre 2015.
\hrule

\section*{Résumé}
\begin{itemize}
  \item Le rapport se \textbf{présente} dans sa conception, son objectivité et ses participants.
  \item Par le passé, il y a déjà eu beaucoup d'\textbf{aides humanitaires en Haïti}, 20 millions d'euros y ont ainsi été investis par des organisations belges. Cependant le gouvernement fédéral agit peu pour Haïti, ce sont principalement les sous-entités politiques (Wallonie, Flandre, communes) qui financent des ONG. Ces investissements se concentrent d'abord sur la sauvegarde de la population, puis sur l'éducation, le développement durable et l'égalité des sexes.
  \item Premier pays noir colonisé indépendant en 1804, Haïti reste sur un modèle oligarchique archaïque jusqu'aux années 1915-1934 durant lesquelles l'occupation du territoire par les \'Etats-Unis mène à de petites avancées tant technologiques que démocratiques. Au milieu du vingtième siècle, une tentative de mouvement social aboutit par un totalitarisme dictatorial de 1957 à 1986. Par la suite le régime tend à se démocratiser mais avec de grosses lacunes (voir point suivant).
  \item La \textbf{situation politique} du pays est grave. La démocratie n'y est que partielle et masque de graves problèmes de corruption. Un manque de coopération inter-ministérielle augmenté par une absence de système hiérarchique au niveau national (les communes sont plus ou moins seules) empêche la conduite de toute politique efficace.
  \item La \textbf{situation géographique} du pays la place sur la même île que la République Dominicaine avec qui elle n'entretient pas de bonnes relations. La République Dominicaine a d'ailleurs déchu de nationalité tous les haïtiens immigrés depuis les années 1940, causant une vague rentrant au pays qu'Haïti n'a pas pu gérer.
  \item La \textbf{situation économique} du pays est problématique. Depuis la fin du maintien de la Gourde (HTG) face au Dollar (USD) en 2015, la Gourde chute. 81\% des foyers sont endettés, un quart des activités professionnelles doivent emprunter pour poursuivre une activité économique. Les fonds alloués sont fortement soupçonnés de corruption par les ONG sur place. Le chômage est à 65\%. 25\% des enfants travaillent, souvent dans des situations de domesticité similaires à un esclavage.
  \item La \textbf{situation environnementale} est catastophique. La déforestation massive du pays cause une érosion des sols causant la perte de pans majeurs de l'île dans la mer chaque année, diminuant les sols arables et causant la disparition d'espèces sur l'île. L'île est également dans une zone à risque (deuxème au ranking mondial) pour les catastrophes naturelles.
  \item La \textbf{situation éducative/culturelle} n'en mène pas large non plus. L'accès à l'éducation est fortement limité, privatisé par manque de fonds gouvernementaux et 80\% des jeunes diplômés quittent le pays dans l'année. La culture haïtienne en fait un peuple fier qui peine à collaborer et causant une désorganisation nationale. Les croyances locales (religion et magie) motivent aussi beaucoup de décisions au détriment de la rationalité.
  \item La \textbf{situation sociale} montre une grande inégalité homme/femmes ainsi qu'un problème sur la traite des enfants. Les enfants de villages pauvres sont parfois envoyés en tant que domestiques dans des familles de la ville. Les lois sociales sont très souvent contournées et les pratiques de certaines entreprises s'approchent de l'esclavage. Les droits sont très peu respectés et le gouvernement ne bouge que peu.
  \item Les \textbf{pouvoirs publics} sont assez similaires aux républiques occidentales dans leur théorie avec une division des pouvoirs ministériels et une décentralisation gouvernementale en départements, secteurs, communes, ... Cependant la réalité est toute autre. Un manque cruel de coopération, de fonds et de cohérence produit un abandon des structures qui sont quasiment toutes obligées de se débrouiller dans leur financement. La majorité des structures de base (santé, eau, alimentation, agriculture, éducation) sont largement supportées (financièrement et en logistique) par des ONG extérieures et locales.
\end{itemize}

\section*{Notes}
Le document est très intéressant dans sa première partie (< pg. 60) et présente de manière exhaustive tous les facteurs pertinents du développement haïtien.
Les différents facteurs sont expliqués par leur histoire, les acteurs qui sont employés dans le secteur concerné et les défis auxquels ils font face.

En résumé de ce document, on pourrait dire que Haïti est un pays qui se veut devenir une république calquée sur le modèle occidental. Cependant la réalité est bien différente puisqu'une profonde différence de mentalité, une corruption latente de la classe politique et une quasi-incapacité des pouvoirs à collaborer.

A la lecture on se rend compte qu'Haïti fait face à des défis à tout point de vue et qu'aucun secteur n'est épargné par les problèmes du pays. La situation est grave et sans les acteurs étrangers le pays serait sans doute voué à sa perte. Cependant une réelle volonté de changer les choses dirige le pays qui s'organise de plus en plus en comités locaux qui prennent les choses en main. Le plus grand défi actuel étant de mettre à l'échelle ces organisations locales et organiser un travail à plus grande échelle.

La seconde partie du document est beaucoup plus technique et vise à présenter tous les axes de résolution des problèmes haïtien pour chaque organisation compétente. Enfin le document se cloture par une vue des synergies possibles entre ces mêmes organisations afin de permettre au pays d'institutionaliser ce qui n'est pour le moment qu'un système d'ONG locales volontaires qui doivent se débrouiller sans aide financière de l'état (et pourtant, ces ONG sont réellement ce qui fait tourner le pays, sans elles la population serait réellement à l'abandon).

On notera enfin une profonde injustice entre le monde urbain qui absorbe la majorité des ressources du pays et le monde rural qui est livré à lui-même par les autorités, d'où la concentration des ONG sur ces mêmes zones.

\section*{Contenu}
Le document comporte dix chapitres qui vont être ci-dessous synthétisés avec une emphase sur le contenu relatif à la problématique de l'eau.
  \subsection*{Chapitre 1 - \'Elaboration de l'analyse}
  L'analyse est collaborative, de nombreuses organisations participent au développement Haïtien. Protos semble avoir un rôle directeur dans cet ensemble. Chaque organisation a son propre domaine de compétence et l'a impliqué dans la rédaction du rapport, Protos s'est chargé de l'accès et assainissement de l'eau. On apprend que le document a été difficile à rédiger à cause d'une certaine fatigue des organisations locales à qui on demanderait trop de travail suite au séisme de 2010, d'une diminution des fonds d'aide (les bailleurs se retirent). Quelques notes sont faites quant aux objectifs généraux de l'ensemble des acteurs (principalement économiques et sociaux).

  \subsection*{Chapitre 2 - Résumé des précédentes actions}
  Plusieurs actions ont été entreprises avant le projet. Le projet de relèvement d'Haïti a démarré depuis 2010 après le séisme, fini en 2014. On priorisait d'abord l'alimentation, l'éducation, la santé et les droits des enfants. Les précédentes actions ont instauré un climat de confiance entre les localités et les acteurs externes. On observe un tableau récapitulatif des acteurs et objectifs de chaque mission humanitaire organisée depuis 2010 ainsi que leurs budgets respectifs (plusieurs millions d'euros).

  Le gouvernement Belge fédéral ne priorise pas la situation en Haïti, en revanche c'est le cas pour  la \textit{Wallonie Bruxelles International} (WBI). L'intervention est essentiellement indirecte, par le biais d'intermédiares tels que l'UNICEF. Le gouvernement flamand se concentre majoritairement sur certains projets de potabilisation de l'eau en Haïti. Certaines communes appuient quant à elle des ONG particulières. Avec 20 millions d'euros investis par le biais d'ONG belges, Haïti est le deuxième pays le plus aidé par la Belgique\footnote{http://www.ong-livreouvert.be} après la RDC.

  Il y a également une certaine coopération internationale même si elle s'avère tendue avec notamment la République Dominicaine (voisin Est), avec qui Haïti a un problème d'immigration, notamment dans le domaine des travailleus saisonniers.

  Les actions se concentraient initialement sur la sauvegarde de la population suite aux catastrophes naturelles, pour ensuite se diversifier. Notamment dans l'inclusion sociale et l'égalité des sexes, ainsi que dans la gestion des futurs désastres (Haïti est dans une zone à risque).

  \subsection*{Chapitre 3 - Situation politique, économique, sociale et environnementale}
  Ce chapitre commence par présenter sous forme de fiche la plupart des indicateurs (démographiques, géographiques, socio-économiques et culturels) [pages 26-28].\bigskip

  \textit{[Le texte suivant est un résumé narratif du texte de 20 pages établissant la situation politique du pays, intéressant à lire de manière contextuelle mais non pertinent dans le contexte du travail]}

  Le contexte politique de l'île se base sur ses racines colonialistes dont il est le premier pays noir à s'être libéré en 1804. Le premier gouvernement était militaro-oligarchique calqué sur la forme coloniale en continuant d'exploiter les plantations en place pour soutenir l'économie en place. La politique instable et le manque d'unité du pays a fait manquer le coche de la mécanisation du pays au XIX\up{ème} siècle.

  De 1915 à 1934, le pays est occupé par les USA. Cette période mène à la création d'une démocratie représentative ainsi qu'une timide industrialisation et urbanisation.

  Cependant l'écart pauvres-riches se creuse et la scission entre l'urbain et le rural s'opère nettement. En 1950 une tentative de socialisation se met en place, réprouvée rapidement par un régime totalitaire de 1957 à 1986 qui marque profondément le pays et renforce la division entre riches et pauvres. Le pays manque une nouvelle fois le coche de la révolution industrielle.

  En 1990, des élections ramènent à une démocratie, balayée en 1991 par un coup d'état qui remet en place une répression des droits sociaux. En 1994, le gouvernement élu est remis en place avec l'aide de l'armée des \'Etats-Unis.

  Ce gouvernement, bien que largement plus démocratique, n'est pas capable d'inverser l'apauvrissement du pays ni de réparer les larges torts institutionnels en place. En 2004, le gouvernement en place est forcé de quitter la direction du pays, remplacé par un gouvernement intérimaire appuyé cette fois par les Nations Unies jusqu'aux élections de 2006.

  En janvier 2010, un grand séisme affaiblit le pays et sonne le retour d'un gouvernement de transition appyé par l'aide internationale. Ce gouvernement de transition est critiqué et toujours présent malgré des votes du Sénat Haïtien pour les déloger. Les organisations haïtiennes souhaitent le départ des organisations étrangères dès 2011, mais l'ONU est toujours présente en 2015.

  Pendant ce temps, les élections sont absentes, les mandats arrivent à terme sans remplacement, des élections en 2015 n'obtiennent que 18\% de participation.

  On constate que la politique Haïtienne est très fortement influencée par des puissances comme les USA, la France, le Canada, l'Espagne et l'UE (bien que plus respectueuse des droits humains comparé aux quatre premières). La corruption est également un facteur de ralentissement avec Haïti qui est le 23\up{ème} pays le plus corrompu au monde\footnote{https://www.ranker.com/list/the-most-corrupt-countries-in-the-world/info-lists}.

  L'\'Etat en place n'a jamais eu de vision au long terme et le séisme de 2010 a détruit les structures en place, sans que l'aide massive de la communauté internationale ne permette une amélioration claire depuis cette catastrophe. Les structures étatiques sont toujours trop centralisées dans la capitale, les zones rurales étant parfois carrément à l'abandon.

  Au niveau social justement, la précarité économique se fait actuellement ressentir avec 81\% des foyers haïtien qui sont endettés. L'incapacité de l'état à fournir des services de base mène à une privatisation de la plupart des secteurs de nécessité basique (90\% des établissements scolaires sont privés, 80\% pour les hopitaux). Les ONG sont les seules structures fournissant les besoins de base à faible coût mais cela n'est pas suffisant. Pire encore, les dépenses pour le social de la part de l'état diminuent, tandis que la Gourde (HTG), monnaie nationale, dégringole face au Dollar depuis 2015.

  Toutes ces précarités mènent le peuple haïtien à de nombreuses grèves parfois violentes.

  L'incapacité de l'état est souvent supplantée au niveau local par des groupes reprenant le rôle du gouvernement, mais ces groupes, bien qu'utiles, peinent à travailler ensemble, augmentant l'incapacité de travail de groupe au niveau national.

  Si le travail manuel est la majeure source de revenus des foyers du pays, la demande est largement inférieure à l'offre et les salaires ne sont pas suffisants, menant parfois même à des situations d'esclavage (y compris la domesticité des enfants).

  L'accès à l'éducation est réduit et l'analphabétisme est élevé (surtout en zone rurale). L'inattractivité du pays cause une grande fuite du capital intellectuel (80\% des haïtiens diplômés quittent le pays l'année de leur graduation).

  Enfin, les croyances socio-culturelles du pays (vaudou, magie, esprits des anciens) influent au quotidien les décisions au détriment de la rationalité des choix.

  Au niveau économique, le système en place relève du catastrophique, les émigrés haïtien produisent 3.2 fois plus de capital que les haïtiens insulaires qui sont pourtant 5 fois plus nombreux. L'économie haïtienne est fortement endettée et la dépendance au dollar jusque 2015 a provoqué un manque cruel de politique économique traduit par une dépréciation massive de la Gourde après 2015.

  L'environnement est problématique en Haïti. Placé deuxième sur la liste des pays les plus à risque, le pays est d'autant plus vulnérable que les politiques en place (déforestation massive, manque d'infrastructures anti-innondations, ...) ne font qu'empirer la situation. La production d'énergie est majoritairement au charbon.
  La déforestation mène à une érosion massive des sols (>50\% du territoire) et une perte annuelle de 40 millions de tonnes de terres arables dans la mer, acrruant l'appauvrissement des sols et les espèces en voie de disparition. La production agricole est en décroissance, les réserves en eau se réduisent également.

  \subsection*{Accès à l'eau, Assainissement et Gestion durable}
  Le taux de couverture est de 64\% au niveau national.
  \begin{itemize}
    \item 77\% zones urbaines
    \item 48\% zones rurales
  \end{itemize}
  Près de 30\% de la population se fournit en eau dans des rivières non-sanitarisées.

  Depuis 2009, le pays a réformé l'institutionalisation du secteur de l'eau. Désormais ce sont les municipalités qui gèrent l'eau, les usagers et comités de gestion sont légalement reconnus, des professions autour du service de l'eau peuvent opérer et le secteur privé se développe.

  Le problème sanitaire est grave en Haïti, 26\% des Haïtiens ont accès à une eau saine. Seul un tiers de la population a accès à des installations sanitaires de base. 41\% de la population rurale pratique la défécation à l'air libre.

  Actuellement, le secteur de l'eau dépend trop de l'aide internationale, le système local est trop jeune et inexpérimenté, l'état ne soutient ni n'aide à la gestion des structures.

  \subsection*{Chapitre 4 - Société civile, autorités décentralisées et pouvoirs publics}
  La volonté d'indépendance décrite plus haut mène à une grande variété d'acteurs de la société civile\footnote{Société civile : qui n'est pas l'état et qui n'est pas lucratif, mais qui défend les intérêts citoyens (e.g.: syndicats, comités, associations, ...)}. On trouve ces acteurs à des degrés de taille et d'organisation très variés.

  Ces acteurs sont importants dans la société haïtienne qui se repose énormément sur leur travail. On constate cependant que ni le gouvernement ni la population ne s'engagent réellement auprès des organismes de la société civile. Le financement vient principalement de fonds d'aide internationaux, aides de pays extérieurs et récoltes auprès des haïtiens (locaux et expatriés).

  La république d'Haïti est démocratique avec une assemblée nationale divisée en deux chambres, une cour de cassation, des cours d'appel, tribunaux, ... En théorie ce gouvernement est décentralisé avec une division en départements, arrondissements, communes et sections communales. Cependant la mise en application se fait attendre et on observe en pratique que seule la commune est une entité fonctionnelle en dehors du gouvernement ministériel. Certaines branches étatiques disposent cependant d'aide décentralisée, notamment le ministère de la Santé qui dispose d'antennes locales, de même que l'Agriculture.

  Le système éducatif est divisé en deux parties ; l'informel qui se charge des savoirs de base (alphabétisation, savoirs techniques) pour les >15 ans et l'informel qui est plus classique (maternel, primaire, secondaire, supérieur). La très grande majorité des établissements scolaires ne dépend pas de l'état.

  Le système de santé est également majoritairement (93\%) financé par les aides non-étatiques et très peu d'investissements sont réalisés dans la santé haïtienne.

  \subsection*{Accès et Assainissement de l'eau}
  C'est le ministère des travaux publics, transports et communications (MTPTC) qui est officiellement chargé de ces responsabilités. Il délègue cela à la Direction Nationale d'Eau Potable et Assainissement (DINEPA), qui se charge de l'eau de population et industrielle.

  La fuite du capital intellectuel Haïtien ainsi que le manque de financement pour les organismes de gestion de l'eau ne permet pas une productivité (ni motivation) suffisante.

  Dans le secteur de l'eau également le manque de cohérence institutionnelle se fait ressentir avec plusieurs centaines de comités d'eau (les CAEP, Comités d'Approvisionnement en Eau Potable) qui gèrent au niveau rural les points d'eau. Il n'y a pas de concertation municipale, départementale ou nationale au niveau de ces comités.

  Les ONG sont très actives dans le secteur de l'eau, surtout en milieu rural et dans les petites villes défavorisées (contrairement à la capitale qui aspire la plupart des fonds du pays).

  \subsection*{Analyse et solutions de la société civile et des administrations}
  La société civile existe donc mais est morcelée et désorganisée. L'égalité des sexes est théoriquement assimilée mais doit encore avancer dans la pratique. Plusieurs ONG extérieures se retirent du pays maintenant que la situaiton grave post-séismique est terminée. Les Haïtien en ont conscience et se mettent en place pour combler les départs par des organisations locales.

  Les succès des organisations locales sont relatifs. Certaines organisations ont beaucoup de succès, mais la plupart restent trop petites et peu performantes, et toujours ce problème de manque de communication et travail d'équipe qui retient les avancées nationales.

  \textit{[Le document liste pour les acteurs les problématiques de chaque secteur et les voies de résolution possibles dans les pages 60 à 87]}

  \subsection*{Analyse des partenaires potentiels}
  \textit{[Le document liste jusqu'en page 102 les acteurs locaux et internationaux présents dans le pays, par domaine d'activité. On constate la forte présence de la Belgique en général (et de Protos)]}

  \subsection*{Modèles de changement}
  \textit{[Le document liste jusqu'en page 119 les objectifs principaux de changement de mentalité en Haïti et motive/donne des pistes pour chacun des modèles. Ces modèles touchent à tous les thèmes précédemment évoqués]}

  \subsection*{Synergies}
  \textit{[Le document présente jusqu'à la page 128 les possibilités de synergies entre les différentes organisations locales et extérieures selon les thématiques et avec de nombreux projets et paramètres (long, court terme, ...). On peut voir cette partie du document comme un très large "qui fait quoi ?"]}
\end{document}
