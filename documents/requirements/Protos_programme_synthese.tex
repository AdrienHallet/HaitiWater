\documentclass[a4paper, 11pt]{article}
\usepackage[french]{babel}
\usepackage[utf8]{inputenc}
\usepackage[T1]{fontenc}
\usepackage{datetime}
\usepackage{float}
\usepackage{eurosym}
\usepackage{comment} % enables the use of multi-line comments (\ifx \fi)
\usepackage{graphicx}

\usepackage{fullpage} % changes the margin
\begin{document}
%Header-Make sure you update this information!!!!
\noindent
\large\textbf{Synthèse "Programme Protos 2017-2021"} \hfill \textbf{Distribution de l'eau en Haïti} \\
\normalsize Deknop Céline \hfill Université Catholique de Louvain \\
Hallet Adrien \hfill \today \\
Strebelle Sébastien

\section*{Abstract}
Synthèse du document \textit{Programme 2017-2018 - Haïti} envoyé par Protos le 05 septembre 2018. Ce document établit les objectifs et le fonctionnement général de Protos durant son programme haïtien, mis en relation avec les objectifs globaux de Protos en général.
\hrule

\section*{Résumé}
\begin{itemize}
  \item Après un bref descriptif global du projet, le document présente les \textbf{partenaires} de Protos.
  \item On aborde ensuite les \textbf{objectifs} de protos ; apporter l'eau et rendre la population autonome. Ces objectifs sont motivés par des \textit{valeurs (équité, etc) et méthodes (impliquer les locaux, etc)}.
  \item Les différents \textbf{acteurs} sont présentés de manière exhaustive, du gouvernement au particulier avec tous les intermédiaires. Chaque acteur dispose d'une fiche contenant ; présentation du groupe, objectifs, risques.
  \item La \textbf{théorie du changement} est exposée par Protos. Concept visant à éduquer la population plutôt que de les lâcher dans un système dont ils ne sauront pas s'occuper.
  \item Exposition qualitative et quantitative des risques et objectifs concrets sous forme de tableaux en cumul sur les cinq années du projet.
  \item \textbf{Motivation du projet} et des choix (géographiques et objectifs), leur pertinence et raisons. Sans doute visant à l'obtention de fonds ou à une justification pour une hiérarchie.
  \item \textbf{Durabilité} à tous points de vue, on notera tout particulièrement la nécessité de résistance aux catastrophes naturelles ainsi que la volonté d'inclure toute la population dans le projet.
  \item Le document analyse la \textbf{capacité des partenaires} à atteindre les objectifs que leur fixe Protos.
  \item Le document termine par une explication de la synergie entre les acteurs externes au pays (en Belgique et ailleurs).
\end{itemize}

\section*{Notes}
Document partiellement intéressant. On note la présence de beaucoup de justification administrative (néanmoins intéressantes pour comprendre les enjeux) sans doute destinée à la hiérarchie. Le document permet de comprendre le mode de fonctionnement très éducatif de Protos qui ne se contente pas de poser des systèmes mais aide réellement les populations à s'en sortir. On voit par ce document que Protos a murement réfléchi ce projet et est documentée. L'organisation montre également une grande capacité de communication en interagissant avec de nombreuses autres entités haïtiennes et extérieures.

Il est à noter que dans la partie des motivations et objectifs concrets, beaucoup de données numériques sont fournies (superficies, nombre de sources, nombre de familles, ...) et pourraient s'avérer utile par la suite.

\section*{Contenu}

\subsection*{Protos en Haïti}
Le document montre bien que Protos est une ONG internationale et que Haïti n'est qu'une branche de leurs activités. Tout au long du document on compare les objectifs et résultats globaux à ceux de Haïti. Protos Haïti suit des directives générales et des principes propres à Protos, mais en appliquant le contexte haïtien à leur mode opératoire.

\subsection*{Localisation}
Protos agit sur deux zones principales nommées par le cours d'eau les traversant :
\begin{description}
  \item[Nord-Ouest] - Région dite "Moustiques"
  \item[Centre-Est] - Région dite "Onde Verte"
\end{description}
Ces zones sont rurales et c'est une volonté de Protos que de s'impliquer là où la communauté internationale a la plus faible présence.

\subsection*{Partenaires}
Protos supervise plusieurs organisations locales. Protos apporte son soutien (expérience, logistique, finances) tandis que les localités tentent de solutionner les problèmes. Chaque zone de travail dispose de sa propre ONG locale, multipliant le nombre de partenaires de Protos. On remarque trois groupes d'acteurs :
\begin{itemize}
  \item Le triangle Autorités locales / Fournisseurs / Usagers
  \item Le gouvernement national
  \item Les autres ONG (locales ou non)
\end{itemize}

Le gouvernement étant instable, les responsabilités sont peu claires et/ou pratiques. Par exemple la gestion de l'eau relève du ministère (désorganisé) et n'impliquera pas les localités avant 2030. A la lecture du document, on comprend que Protos cherche à être le plus inclusif possible et à impliquer un maximum d'acteurs dans ce projet. Protos se repose beaucoup sur les entités locales, notamment les services étatiques (y compris l'université).
\subsubsection*{Risques}
Protos identifie plusieurs risques possibles pouvant nuire au bon déroulement du projet :
\begin{itemize}
  \item Instabilité et manque d'impartialité des politiques
  \item Népotisme dans l'attribution des emplois au détriment de réelles compétences, creusant d'autant plus le manque actuel de personnel qualifié.
  \item Favoritisme dans l'accès à l'eau par les autorités, principalement pour les industriels.
  \item Limite des moyens causant le ralentissement voire l'arrêt total du travail (grèves fréquentes).
\end{itemize}

D'autres risques indépendants des localités sont exposés (catastrophe naturelle, crise économique).

\subsection*{Objectifs}
Le projet de Protos est d'apporter l'eau à tous. Cela comprend toutes les filières de l'eau ; transport, assainissement, eau de consommation, irrigation, industrielle, ainsi que l'eau nécessaire pour la nature. Sur Haïti ces principes sont conservés, mais l'accent est mis sur l'accès à l'eau potable et son assainissement pour les populations. Le document insiste sur la nécessité d'impliquer les acteurs locaux et souligne l'absence d'organisation gouvernementale gérant ces problèmes.

Protos souhaite également protéger les bassins afin de permettre une meilleure rétention (et donc accessibilité) à l'eau. Le déboisement de l'île a causé un fort problème d'érosion des sols. L'objectif sera d'améliorer (ou au moins maintenir) le cycle de l'eau, tenter de diminuer les inondations. De cet objectif découle plusieurs sous-objectifs connexes ; meilleure agriculture, santé, ...

A terme, Protos souhaiterait que les deux régions impliquées servent d'exemple à plus grande échelle pour répandre les avancées à travers le pays. On voit tout au long du document qu'ils travaillent dans une optique hierarchique "bottom-up" où la conscientisation et apport en eau aux populations devrait progressivement monter jusqu'à l'état pour avoir ensuite un système fonctionnel.

\subsection*{Théorie du Changement - TDC}
Protos souhaite éduquer la population. Pour ce faire ils impliquent les localités et tentent de conscientiser tout le monde au problème de l'eau. Dans sa théorie du changement, Protos mentionne trois \textbf{sphères} :
\begin{description}
  \item[Contrôle] - En améliorant le travail sur le terrain, on conscientise les acteurs locaux. Pour ce faire, on améliore la gestion et la coordination, les échanges entre les acteurs, on éduque les acteurs.
  \item[Influence] - Une fois le contrôle établi, les acteurs locaux s'impliquent dans la gestion de l'eau et aident à tout point de vue (matériel, financier, social) pour influencer la totalité de la population et espérer une conscientisation globale.
  \item[Impact] - Ce sont les résultats visés par Protos, on peut les résumer par \textit{un accès durable, équitable et indépendante à l'eau potable pour tous}.
\end{description}

Cette théorie du changement se veut progressive et transmissible dans la population par tous (Protos initie le changement pour que la population prenne le relais).

\subsection*{Durabilité}
\begin{description}
  \item[Technique] Le projet devrait être résistant aux cyclones et séismes, les locaux auront les capacités de maintenir et réparer le système.
  \item[Economique] Permettre une viabilité économique en rendant le système de gestion performant (éducation) et en augmentant le recouvrement des redevances des usagers (factures).
  \item[Sociale] Inclure tout le monde dans le projet, peu importe l'âge et le sexe.
  \item[Institutionnelle] Augmenter la collaboration et le bon fonctionnement des structures.
  \item[Environnementale] Protection des sols et des sources.
\end{description}
\end{document}
